\chapter{文件系统}

\section{文件操作}

\subsection{打开文件}

计算机除了拥有计算的能力之外,一定要有数据的存储能力,而对于数据的存储一般就可以通过文件的形式来完成。在Python中直接提供有文件的I/O(Input/Output)处理函数操作,利用这些函数可以方便地实现读取和写入。 \\

\lstinline|open()|函数的功能是进行文件的打开,在进行文件打开的时候如果不设置任何的模式类型,则默认为r(只读模式)。

\begin{lstlisting}[language=Python]
def open(
    file, mode='r', buffering=None, encoding=None,
    errors=None, newline=None, closefd=True
)
\end{lstlisting}

\begin{table}[H]
	\centering
	\setlength{\tabcolsep}{5mm}{
		\begin{tabular}{|c|l|}
			\hline
			\textbf{模式} & \textbf{描述}                                  \\
			\hline
			r             & 使用只读模式打开文件,此为默认模式             \\
			\hline
			w             & 写模式,如果文件存在则覆盖,文件不存在则创建   \\
			\hline
			x             & 写模式,新建一个文件,如果该文件已存在则会报错 \\
			\hline
			a             & 内容追加模式                                   \\
			\hline
			b             & 二进制模式                                     \\
			\hline
			t             & 文本模式(默认)                               \\
			\hline
			+             & 打开一个文件进行更新(可读可写)               \\
			\hline
		\end{tabular}
	}
	\caption{文件打开模式}
\end{table}

如果以只读的模式打开文件,并且文件路径不存在的话,就会出现FileNotFoundError`错误信息。

\begin{lstlisting}[language=Python, title=文件操作]
def main():
    file = open(file="data.txt", mode="w")
    print("文件名称:%s" % file.name)
    print("访问模式:%s" % file.mode)
    print("文件状态:%s" % file.closed)
    print("关闭文件...")
    file.close()
    print("文件状态:%s" % file.closed)

if __name__ == "__main__":
    main()
\end{lstlisting}

\begin{tcolorbox}
	\mybox{运行结果} \\
	文件名称:data.txt \\
	访问模式:w \\
	文件状态:False \\
	关闭文件... \\
	文件状态:True
\end{tcolorbox}

\subsection{文件读写}

当使用\lstinline|open()|打开一个文件之后,接下来可以使用创建的文件对象进行读写操作。 \\

使用读模式打开文件后,可以使用循环读取每一行的数据内容。Python在进行文件读取操作的时候也可以进一步简化操作。文件对象本身是可以迭代的,在迭代的时候是以换行符进行分割,每次迭代就读取到一行数据内容。

\begin{table}[H]
	\centering
	\setlength{\tabcolsep}{5mm}{
		\begin{tabular}{|l|l|}
			\hline
			\textbf{方法}                             & \textbf{描述}          \\
			\hline
			def close(self)                           & 关闭文件资源           \\
			\hline
			def fileno(self)                          & 获取文件描述符         \\
			\hline
			def flush(self)                           & 强制刷新缓冲区         \\
			\hline
			def read(self, n: int = -1)               & 数据读取,默认读取全部 \\
			\hline
			def readlines(self, hint: int = -1)       & 读取所有数据行         \\
			\hline
			def readline(self, limit: int = -1)       & 读取每行数据           \\
			\hline
			def truncate(self, size: int = None)      & 文件截取               \\
			\hline
			def writable(self)                        & 判断文件是否可以写入   \\
			\hline
			def write(self, s: AnyStr)                & 文件写入               \\
			\hline
			def writelines(self, lines, List[AnyStr]) & 写入一组数据           \\
			\hline
		\end{tabular}
	}
	\caption{文件读写方法}
\end{table}

既然所有的文件对象最终都需要被开发者关闭,那么可以结合with语句实现自动的关闭处理。通过with实现所有资源对象的连接和释放是在Python中编写资源操作的重要技术手段,通过这样的操作可以极大地减少和优化代码结构。 \\

\begin{lstlisting}[language=Python, title=读取文件]
def main():
    with open(file="data.txt", mode="r", encoding="utf-8") as file:
        for line in file:
            print(line, end='')

if __name__ == "__main__":
    main()
\end{lstlisting}

\begin{lstlisting}[language=Python, title=data.txt]
小灰	16
小白	17
小黄	21
\end{lstlisting}

\begin{tcolorbox}
	\mybox{运行结果} \\
	小灰	16 \\
    小白	17 \\
    小黄	21
\end{tcolorbox}

\begin{lstlisting}[language=Python, title=写入文件]
def main():
    with open(file="data.txt", mode="w", encoding="utf-8") as file:
        info = {"小灰": 16, "小白": 17, "小黄": 21}
        for name, age in info.items():
            file.write("%s\t%d\n" % (name, age))

if __name__ == "__main__":
    main()
\end{lstlisting}

\begin{tcolorbox}
    \mybox{运行结果} \\
    \textbf{data.txt} \\
    小灰	16 \\
    小白	17 \\
    小黄	21
\end{tcolorbox}

\newpage

\section{文件缓冲}

\subsection{文件缓冲}

在使用\lstinline|open()|创建一个文件对象的时候,默认情况下是不会启用缓冲的。在\lstinline|open()|中提供的buffering参数描述的就是文件处理缓冲的定义,在进行文件写入的时候利用缓冲可以避免频繁的I/O资源占用。 \\

开启缓冲可以提高写入效率,buffering参数设置有3种类型:

\begin{enumerate}
    \item 全缓冲(buffering > 1):当标准I/O缓存被填满后才会进行真正I/O操作,全缓冲的典型代表就是对磁盘文件的读写操作。
    
    \item 行缓冲(buffering = 1):在I/O操作中遇见换行符时才执行真正的I/O操作,例如在使用网络聊天工具时所编辑的文字在没有发送前是不会进行I/O操作的。
    
    \item 不缓冲(buffering = 0):直接进行终端设备的I/O操作,数据不经过缓冲保存。
\end{enumerate}

如果想要观察到行缓冲的使用特点,就不能直接使用with语句,因为with最后会执行\lstinline|close()|的关闭操作,而一旦关闭,则缓冲的内容会全部进行输出。 \\

使用\lstinline|flush()|进行缓冲区的强制清空,一旦强制清空之后,缓冲区的内容将全部输出。每次使用\lstinline|close()|关闭文件流的时候默认情况下也会调用\lstinline|flush()|进行缓冲区的清空处理。

\begin{lstlisting}[language=Python, title=文件缓冲]
import os

def main():
    file = open(file="data.txt", mode="w",
                encoding="utf-8", buffering=1)
    file.write("This is a test.")
    os.system("pause")  # 程序暂停
    file.flush()        # 强制清空缓冲区
    os.system("pause")  # 程序暂停
    file.close()

if __name__ == "__main__":
    main()
\end{lstlisting}
    
\begin{tcolorbox}
    \mybox{运行结果} \\
    \textbf{data.txt} \\
    This is a test.
\end{tcolorbox}

\newpage

\section{文件系统}

\subsection{文件系统}

文件系统是OS中负责管理持久数据的子系统,也就是负责把用户的文件存到磁盘硬件中,即使计算机断电了,磁盘里的数据并不会丢失,所以可以持久化地保存文件。文件系统的基本数据单位是文件,它的目的是对磁盘上的文件进行组织管理。 \\

Linux最经典的一句话是``一切皆文件”,不仅普通的文件和目录,就连块设备、管道、socket等,也都是统一交给文件系统管理的。 \\

文件和目录是按照层次关系管理的,这种结构可以用树表示。文件夹中包含了子文件夹或者文件,这样就形成了一个树的结构。不过也有一些特殊情况,例如团队在协同工作时会共享一些信息,两个人有着不同的目录,但是可以共享同一份文件。 \\

Linux文件系统会为每个文件分配索引结点(inode)和目录项(directory entry)这两个数据结构。索引结点用来记录文件的元信息,比如inode编号、文件大小、访问权限、创建时间、修改时间、数据在磁盘的位置等。目录项用来记录文件的名字、索引结点指针以及与其它目录项的层级关联关系。多个目录项关联起来,就会形成目录结构,但它与索引结点不同的是,目录项是由内核维护的一个数据结构,不存放于磁盘,而是缓存在内存。 \\

由于索引结点唯一标识一个文件,而目录项记录着文件名,所以目录项和索引结点的关系是多对一。也就是说,一个文件可以有多个别名,例如,硬链接的实现就是多个目录项中的索引结点指向同一个文件。

\begin{figure}[H]
	\centering
	\includegraphics[scale=0.35]{img/C5/5-3/1.png}
	\caption{文件系统}
\end{figure}

\subsection{硬链接与软链接}

链接简单说实际上是一种文件共享的方式,主流文件系统都支持链接文件。链接简单地理解为Windows中常见的快捷方式,Linux中常用它来解决一些库版本的问题,通常也会将一些目录层次较深的文件链接到一个更易访问的目录中。 \\

链接分为硬链接(hard link)和软链接(symbolic link)。从使用的角度讲,两者没有任何区别,都与正常的文件访问方式一样,支持读写,如果是可执行文件的话也可以直接执行。 \\

硬链接通过同一个inode指向原始文件,软链接通过硬盘媒介中的描述指向原始文件。但是,当原始文件的位置发生改变后,inode不会改变,所以硬链接还是正确的,而软链接就无法访问了。

\begin{figure}[H]
	\centering
	\begin{tikzpicture}[scale=0.7]
		\draw[rounded corners] (0,0) rectangle (4,2);
		\draw[rounded corners] (6,0) rectangle (10,2);
		\draw[rounded corners] (12,0) rectangle (16,2);

		\draw (2,1) node {原始文件file1};
		\draw (8,1) node {硬链接file2};
		\draw (14,1) node {软连接file3};

		\draw[rounded corners] (2,-5) circle (2);
		\draw[rounded corners] (14,-5) circle (2);

		\draw (2,-4.5) node {inode};
		\draw (14,-4.5) node {inode};
		\draw (2,-5.5) node {37259183};
		\draw (14,-5.5) node {13549669};
	\end{tikzpicture}
	\caption{硬链接与软链接}
\end{figure}