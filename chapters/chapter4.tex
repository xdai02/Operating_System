\chapter{设备管理}

\section{I/O重定向}

\subsection{标准I/O}

程序对读入的数据进行处理,再输出数据。数据的输入(input)和输出(output)简称为I/O,在没有指定输入输出的情况下,默认为标准输入和标准输出。 \\

打开的文件都有一个文件描述符(fd, file descriptor),表现为一个数字:

\begin{itemize}
    \item 标准输入stdin(键盘):fd = 0
    \item 标准输出stdout(显示器):fd = 1
    \item 标准错误输出stderr(显示器):fd = 2
\end{itemize}

\begin{lstlisting}[language=C, title=标准I/O]
#include <stdio.h>

int main(int argc, char *argv[]) {
    int num;
    printf("(stdin) enter an integer: ");
    fscanf(stdin, "%d", &num);
    fprintf(stdout, "(stdout) num = %d\n", num);
    fprintf(stderr, "(stderr) This is an error message.\n");
    return 0;
}
\end{lstlisting}

\begin{tcolorbox}
    \mybox{运行结果} \\
    (stdin) enter an integer: 123 \\
    (stdout) num = 123 \\
    (stderr) This is an error message.
\end{tcolorbox}

\begin{lstlisting}[language=C, title=文件I/O]
#include <stdio.h>
#include <stdlib.h>

void writeFile(const char *filename) {
    FILE *fp = fopen(filename, "w");
    if(!fp) {
        fprintf(stderr, "File open failed.\n");
        exit(1);
    }
    char *name = "小灰";
    int age = 17;
    double height = 182.3;
    fprintf(fp, "姓名:%s\n年龄:%d\n身高:%.2f\n", 
                name, age, height);
    fclose(fp);
}

void readFile(const char *filename) {
    FILE *fp = fopen(filename, "r");
    if(!fp) {
        fprintf(stderr, "File open failed.\n");
        exit(1);
    }
    char name[32];
    int age;
    double height;
    fscanf(fp, "姓名:%s\n年龄:%d\n身高:%lf\n", 
                name, &age, &height);
    printf("name: %s\n", name);
    printf("age: %d\n", age);
    printf("height: %.2f\n", height);
    fclose(fp);
}

int main(int argc, char *argv[]) {
    const char *filename = "info.txt";
    writeFile(filename);
    readFile(filename);
    return 0;
}
\end{lstlisting}

\begin{tcolorbox}
    \mybox{运行结果} \\
    name: 小灰 \\
    age: 17 \\
    height: 182.30
\end{tcolorbox}

\subsection{I/O重定向(I/O Redirection)}

I/O重定向就是改变标准输入与输出的默认位置。标准输入默认是键盘,通过改成其它输入,就是输入重定向,例如从文本文件里输入。标准输出默认是显示器,通过改成其它输出,就是输出重定向,例如输出到文件。 \\

输出重定向用\lstinline|>|表示,若文件不存在,则创建;若文件已存在,则覆盖。使用\lstinline|>>|时若文件不存在,则创建,若文件已存在,则追加。错误输出重定向用\lstinline|2>|和\lstinline|2>>|表示。 \\

输入重定向用\lstinline|<|表示,但是在输入重定向中\lstinline|<<|可不是表示输入追加。

\begin{lstlisting}[language=C, title=I/O重定向]
#include <stdio.h>

int main(int argc, char *argv[]) {
    int data;
    scanf("%d", &data);
    printf("data = %d\n", data);
    return 0;
}
\end{lstlisting}

\begin{lstlisting}[title=input.txt]
12345
\end{lstlisting}

\begin{lstlisting}[title=编译]
gcc -Wall io_redirection.c -o io_redirection
\end{lstlisting}

\begin{lstlisting}[title=运行]
./io_redirection < input.txt > output.txt
\end{lstlisting}

\begin{tcolorbox}
    \mybox{运行结果} \\
    \textbf{output.txt} \\
    num = 12345
\end{tcolorbox}

\newpage

\section{I/O控制方式}

\subsection{I/O设备}

现代计算机系统总是配有各种类型的外部设备,除了显示器、键盘、打印机、磁带、磁盘外,又出现了光盘、绘图仪、图形数字化仪、鼠标器、激光打印机、声音输入输出设备等,种类繁多。不同外设之间的差异较大,因此I/O性能经常成为系统的瓶颈。 \\

操作系统设备管理的目标包括:

\begin{enumerate}
    \item 向用户提供使用外部设备的方便、统一的接口,按照用户的要求和设备的类型,控制设备工作,完成用户的输入输出请求。方便是指用户能独立于具体设备的复杂物理特性而方便地使用设备;统一是指对不同设备尽量能统一操作方式。方便和统一要求对用户屏蔽实现具体设备I/O操作的细节,呈现给用户的是一种性能理想化的、操作简便的逻辑设备。系统的这种性能亦称为设备的独立性(设备无关性)。
    
    \item 充分利用中断技术、通道技术和缓冲技术,提高CPU与设备、设备与设备间的并行工作能力,充分利用设备资源,提高外部设备的使用效率。
    
    \item 设备管理就是要保证在多道程序环境下,当多个进程竞争使用设备时,按照一定的策略分配和管理设备,以使系统能有条不紊地工作。
\end{enumerate}

\subsection{I/O控制方式}

早期,计算机设计者没有将CPU的执行与I/O操作分开,甚至大多数人认为输入输出与计算的时间应该是同一数量级。后来,他们意识到,CPU要比I/O操作速度高几个数量级。于是,硬件和软件设计师开始寻找一种技术,使CPU计算可以不用等待I/O操作而持续执行。 \\

I/O控制方式包括:

\begin{enumerate}
    \item 程序控制I/O(Programmed I/O):处理器代表一个进程给I/O模块发送一个I/O命令,该进程进入忙等待(busy waiting),直到操作完成。
    
    \item 中断驱动I/O(Interrupt I/O):处理器代表进程向I/O发送命令,如果来自进程的I/O指令是非阻塞的,那么处理器继续执行进程的后续指令。如果I/O指令时阻塞的,那么处理器执行来自操作系统的指令,将当前进程设置为阻塞态并且调度其它进程。
    
    \item 直接存储器访问(DMA):DMA模块控制内存和I/O模块之间的数据交换。为传送一块数据,处理器给DMA模块发送请求,只有当整个数据块传送结束后,它才能被中断。
\end{enumerate}

\newpage

\section{I/O缓冲}

\subsection{I/O缓冲(I/O Buffering)}

在设备管理中,为了缓和CPU与I/O设备速度不匹配的矛盾,提高CPU与I/O设备的并行性、减少对CPU的中断频率,在I/O设备与处理机交换数据时都用到了缓冲区。 \\

缓冲的种类包括:

\begin{enumerate}
    \item 单缓冲:输入时通道先将数据送入缓冲区,CPU从缓冲区取数据处理。通道再送入后续数据,如此反复直到输入完成。输出情形正好相反。由于缓冲区属于互斥区,所以单缓冲并不能明显改善CPU与外部设备的并行性。
\end{enumerate}